\documentclass[conference]{IEEEtran}
\IEEEoverridecommandlockouts

% Packages
\usepackage{cite}
\usepackage{amsmath,amssymb,amsfonts}
\usepackage{algorithmic}
\usepackage{graphicx}
\usepackage{textcomp}
\usepackage{xcolor}
\usepackage{hyperref}

% Custom commands
\def\BibTeX{{\rm B\kern-.05em{\sc i\kern-.025em b}\kern-.08em
    T\kern-.1667em\lower.7ex\hbox{E}\kern-.125emX}}

\begin{document}

\title{Multi-Agent Reinforcement Learning for Coordinated Drone Search and Rescue Operations}

\author{\IEEEauthorblockN{Author Names Here}
\IEEEauthorblockA{\textit{Institution/Department} \\
\textit{University Name}\\
City, Country \\
email@domain.com}}

\maketitle

\begin{abstract}
% TODO: Write abstract (150-200 words summarizing problem, approach, and key findings)
\end{abstract}

\begin{IEEEkeywords}
Multi-agent reinforcement learning, drone swarms, search and rescue, coordination, decentralized control
\end{IEEEkeywords}

\section{Introduction}
% Goal: Motivate the problem and state contributions clearly (½–1 page)

\subsection{Context}
% TODO: Briefly explain why coordinated drone search in SAR is important

\subsection{Problem Gap}
% TODO: Traditional centralized control vs decentralized swarm intelligence; why MARL could help

\subsection{Research Question}
% TODO: Explicitly state your research question

\subsection{Objectives}
% TODO: List your 3–4 research objectives concisely

\subsection{Contributions}
% TODO: Bulleted list of contributions, e.g.:
% \begin{itemize}
%     \item Developed a reproducible MARL simulation framework for drone swarms in SAR
%     \item Compared single-agent RL baseline vs multi-agent RL
%     \item Quantitatively evaluated swarm cohesion, coverage, and robustness under noise and failures
% \end{itemize}

\section{Related Work}
% Goal: Situate your work in context (½ page)

\subsection{Prior Work on Swarm Robotics and SAR Optimization}
% TODO: Brief overview of existing swarm robotics work

\subsection{Existing MARL Frameworks}
% TODO: Mention PettingZoo, MAgent, MADDPG papers, etc.

\subsection{Research Gaps}
% TODO: Lack of open-source, reproducible frameworks or realistic environmental constraints

\section{Methodology}
% Heart of the paper (2–2.5 pages)

\subsection{Simulation Framework Design}
% TODO: Overview of environment(s): grid layout, agent sensors, victim placement, obstacles
% TODO: Constraints modeled (communication loss, sensor noise, energy limits)
% TODO: Architecture diagram or figure of environment setup
% TODO: How agents communicate (shared rewards, message passing, independent learners)

\subsection{Individual Environments}
% Each teammate adds their environment subsection

\subsubsection{Environment A: [Name]}
% TODO: 1 paragraph on environment setup and RL formulation (state, action, reward)

\subsubsection{Environment B: [Name]}
% TODO: 1 paragraph on environment setup and RL formulation (state, action, reward)

\subsubsection{Environment C: [Name]}
% TODO: 1 paragraph on environment setup and RL formulation (state, action, reward)

\subsection{Reinforcement Learning Algorithms}

\subsubsection{Single-Agent Baseline}
% TODO: Describe the policy and objective (e.g., PPO, DQN)

\subsubsection{Multi-Agent RL}
% TODO: Explain how coordination is achieved (e.g., MADDPG, QMIX, MAPPO)
% TODO: Training details (episodes, reward shaping, exploration strategy)
% TODO: Include figure showing training setup or algorithm pipeline

\section{Results and Analysis}
% 1.5–2 pages

\subsection{Performance Metrics}
% TODO: Define metrics
% \begin{itemize}
%     \item Coverage rate
%     \item Time to detection
%     \item Swarm cohesion (e.g., average inter-agent distance)
%     \item Robustness (performance under noise/failure)
% \end{itemize}

\subsection{Quantitative Results}
% TODO: Comparison tables or graphs for Single-Agent vs Multi-Agent results
% TODO: Highlight improvements in coverage/time to detection

% Example table structure:
% \begin{table}[htbp]
% \caption{Performance Comparison}
% \begin{center}
% \begin{tabular}{|c|c|c|c|}
% \hline
% \textbf{Environment} & \textbf{Single-Agent} & \textbf{Multi-Agent} & \textbf{Improvement} \\
% \hline
% Environment A & - & - & - \\
% Environment B & - & - & - \\
% Environment C & - & - & - \\
% \hline
% \end{tabular}
% \label{tab:results}
% \end{center}
% \end{table}

\subsection{Qualitative Observations}
% TODO: Visuals or heatmaps showing swarm paths
% TODO: Notable emergent behaviors (adaptive formation, division of labor)

% Example figure:
% \begin{figure}[htbp]
% \centerline{\includegraphics[width=\columnwidth]{figures/placeholder.png}}
% \caption{Swarm coordination visualization}
% \label{fig:swarm}
% \end{figure}

\section{Discussion}
% ½ page

% TODO: What patterns or strategies did agents learn?
% TODO: Did different environments encourage different emergent behaviors?
% TODO: Where did MARL underperform (e.g., unstable coordination, convergence issues)?
% TODO: How does this contribute to future SAR swarm research?

\section{Conclusion and Future Work}
% ¼–½ page

\subsection{Summary of Findings}
% TODO: Main findings, e.g., "Multi-agent reinforcement learning improved coverage by X% 
% and maintained swarm cohesion despite communication loss"

\subsection{Future Directions}
% TODO: Physical drone deployment, curriculum learning, hybrid communication policies, 
% domain randomization for transfer to reality

\begin{thebibliography}{00}
% TODO: Add 10-15 key sources
% \bibitem{b1} Author, ``Title,'' Journal, vol., no., pp., year.
% \bibitem{b2} Author, ``Title,'' Conference, year.
\end{thebibliography}

\end{document}
